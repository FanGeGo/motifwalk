% Coding: utf-8
% Filename: aaai17_mage.tex
% Description: Paper submit to AAAI'17
% v0.0: File created. Motif-Aware Graph Embedding

% Standard settings by AAAI'17 authors' kit
\documentclass[letterpaper]{article}
\usepackage{aaai17}
\usepackage{times}
\usepackage{helvet}
\usepackage{courier}
\setlength{\pdfpagewidth}{8.5in}
\setlength{\pdfpageheight}{11in}
% PDF metadata
\pdfinfo{
    /Title (Motif-Aware Graph Embedding)
    /Author (Hoang Nguyen, Shun Nukui, Tsuyoshi Murata)
    /Keywords (Graph latent features, vertex representation, skipgram, graph motif)
}
% Section number
\setcounter{secnumdepth}{0}
% Title and authors
\title{Motif-Aware Graph Embedding}
\author{
    Hoang Nguyen \\
    Tokyo Institute of Technology \\
    hoangnt@net.c.titech.ac.jp \\
    \And 
    Shun Nukui \\ 
    Tokyo Institute of Techonology \\
    nukui.s@net.c.titech.ac.jp 
    \And 
    Tsuyoshi Murata \\
    Tokyo Institute of Techonology \\
    murata@c.titech.ac.jp 
}

%% Paper content
\begin{document}
    \maketitle

    \section{Abstract}
        Latent vector representations quality of skipgram model is extremely sensitive to 
        graph context generation method. Generally, random walk is the most popular method
        to create an artificial context from graph. Recent researches in skipgram-based 
        graph embedding have hinted that embedding quality can be improved by manipulating
        the graph context generation process. In this paper, we propose a novel graph embedding
        algorithm that emphasizes motif structure of a graph. Context generation by random walk
        is a short of 1-D mapping from 2-D representation of the graph.

    \setcounter{secnumdepth}{2}
    \section{Introduction}
        \subsection{La}

\end{document}
